\documentclass{IEEEtran}
\pagenumbering{gobble}

\usepackage[english]{babel}
\usepackage{blindtext}
\usepackage{multirow}
\usepackage{graphicx}

\title{\textbf{NL2code}}

% Nischal Mahaveer Chand
% Varun Sundar Rabindranath
% Sai Krishna Karanan
\author{
    Sai Krishna Karanam 
    \texttt{karanam.s@husky.neu.edu}
    \and \\
    Nischal Mahaveer Chand 
    \texttt{mahaveerchand.n@husky.neu.edu}
    \and \\
    Varun Sundar Rabindranath 
    \texttt{rabindranath.v@husky.neu.edu}
}
\date{}

\begin{document}

    \maketitle

    \section{Introduction}
    \blindtext

    \section{Related Word}
    \blindtext

    \section{Dataset}
      \subsection{Standard Datasets}
        \blindtext
        \subsubsection{Django}
        \blindtext
        \subsubsection{HS}
        \blindtext
      \subsection{Our dataset}
      \blindtext

    \section{Models}
    We describe four models, each is build upon Pengcheng's model, but encodes the input in a 
    different method. First we start by briefly describing Pengcheng's model.

      \subsection{Pengcheng's model}
      Pengcheng recognized that adding syntax information to the model would give better
      predictions results XXX. His model, follows an encoder-decoder architecture, and takes
      the raw comment as input and generates an Abstract Syntax Tree (AST) of the corresponding 
      code as output. \\
      \hspace*{4mm}The encoder comprises of an embedding layer and a Bidirectional LSTM (BiLSTM)
      layer. It takes a comment as input, embeds each word in the comment to give 
      $ TE_i $, for each word $ i $ in the comment. Each $ TE_t $ is sequentially feed into the
      BiLSTM layer, to produces a Query Embedding (QE) of 128 dimensions. QE is passed to the
      decoder module. \\
      \hspace*{4mm}The decoder is slightly complicated and works in mysterious ways! Lord 
      Voldomort himself
      blessed it with his divine wand to produce a magical black box, that generates
      AST's! \\

      \subsection{Our models}
      As described, the decoder is already at a state-of-the-art level XXX, and needs no further 
      modifications. All models described hereon use the same decoder architecture as Pengcheng's 
      model with modifications to the encoder and the input data XXX. All models are trained and 
      tested using the dataset decribed in SECTION. % todo add information about the embeddings

        \subsubsection{Basic Concat (BC)}
        For our first attemp to incorporate syntax information into the encoder, we decided to 
        add (append) the POS and phrase ID of each token to the corresponding token embedding, 
        giving us the Augmented Token Embedding (ATE). The ATE is then feed into a modified BiLSTM
        layer that took 130 dimension embeddings, rather than the specified 128 dimensions. \\

        \hspace*{-4mm}Token embedding dimensions: 128 \\
        POS and Phrase ID dimensions: 1 each; total 2 \\
        Augmented token embedding dimensions: 130 \\

        \subsubsection{Linear Projection (LP)}
        To add some syntactic information over a sequence of tokens, we used an embedding layer for
        POS and phrase tags. The resulting TE, POS embedding (POSE), and Phrase embedding (PhE) are
        then concatenated to produce the ATE; which is a (128 * 3) dimension vector. We then
        apply a linear projection (using a dense layer with (128 * 3) input nodes, 128 output 
        nodes, and the linear actiation function). This new vector is passed to the BiLSTM of 
        Pengcheng's model. \\

        \hspace*{-4mm}Token embedding dimension: 128 \\
        POS embedding dimension: 128 \\
        Phrase embedding dimension: 128 \\
        Dense = (128 * 3) input nodes, 128 output nodes \\ 
        ATE = [TE : POSE : PhE] (: is concatenation)\\
        ATE' = Dense(ATE) \\
        QE = BiLSTM(ATE`) \\

        \subsubsection{Linear Projection Reduced Dimension (LPrd)} 
        Subsequently, we noticed that the POS and Phrase vocabulary sizes were relatively smaller
        than token vocabulary size. To avoid redundancy XXX, we changed the embedding dimensions
        of POSE and PhE to 8 and 32 respectively. The process described in LP is then repeated. \\

        \hspace*{-4mm}Token embedding dimension: 128 \\
        POS embedding dimension: 8 \\
        Phrase embedding dimension: 32 \\
        Dense = (128 + 8 + 32) input nodes, 128 output nodes \\
        ATE = [TE : POSE : PhE] (: is concatenation)\\
        ATE' = Dense(ATE) \\
        Query Embedding = BiLSTM(ATE') \\

        \subsubsection{Raw Query Independent Preprojection (AdvLP)}
        Rather than applying one linear projection on ATE, we apply two here, where the first is 
        independent of the input query. \\

        \hspace*{-4mm}Token embedding dimension: 128 \\
        POS embedding dimension: 128 \\
        Phrase embedding dimension: 128 \\
        preprojector = (128 * 2) input nodes, 128 output nodes \\
        Dense = (128 * 3) input nodes, 128 output nodes \\ 
        AE = [POSE : PhE] \\
        PAE = Preprojector(AE) \\
        ATE = [TE : PAE] \\
        ATE' = Dense(ATE) \\
        QE = BiLSTM(ATE') \\
      % end of Models

    \blindtext

    \section{Experiments}
    \blindtext

    \section{Results}
    \resizebox{8cm}{!} {
      \begin{tabular}{ | c | c || c | c | }
        \hline
        \multicolumn{2}{ | c ||}{\multirow{2}{*}{Models}} & \multicolumn{2}{ c | }{Metrics} \\
        \cline{3-4}
        \multicolumn{2}{ | c ||}{} & BLUE & Accu. \\
        \hline
        \multicolumn{2}{ | c ||}{Pengcheng} & 84.5 & 71.6 \\
        \multicolumn{2}{ | c ||}{Base} & 73.2 & 67.9 \\
        \hline
        \multirow{4}{*}{NL2code} & BC & 73.6 & 69.4 \\
        & LP & \underline{74.3} & \underline{69.7} \\
        & LPrd & 73.6 & 69.0 \\
        & AdvLP & 73.7 & 69.1 \\
        \hline  
      \end{tabular} }

    \section{Conclusion}
    \blindtext

    % \bibliographystyle{ieeetran}
    % \bibliography{report}

\end{document}
