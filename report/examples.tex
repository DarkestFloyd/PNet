\documentclass{article}
\usepackage[margin=1in]{geometry}
\pagenumbering{gobble}
\usepackage{multirow}

\begin{document}

  \section{Correctly Predicted Examples}

  \begin{tabular}{ l p{0.8\textwidth}}
    \hline
		\textbf{input} & \texttt{call the method CLASS\_0 . FUNC\_0 with an argument boolean True .} \\
    \textbf{pred.} & \texttt{CLASS\_0.FUNC\_0(True)} \\
    \hline
		\textbf{input} & \texttt{ANY\_1 is an string STR\_0 formatted with CLASS\_0 . ANY\_0 and 
                      result of the method CLASS\_0 . FUNC\_0 , respectively .}	\\
		\textbf{pref.} & \texttt{ANY\_1 = 'STR\_0' \% (CLASS\_0.ANY\_0, CLASS\_0.FUNC\_0())} \\
    \hline
    \textbf{input} & \texttt{remove last element for ANY\_0 .} \\
    \textbf{pred.} & \texttt{ANY\_0 = ANY\_0[:-1]} \\
    \textbf{remark}& \texttt{The fact that the model learned to map the token ``last'' in the 
                     query to ``-1'' in code, tells us that the model has learnt to translate some 
                     semantics in the input query to code.} \\
    \hline
  \end{tabular}

  \section{Incorrectly Predicted Examples}

  \begin{tabular}{ l p{0.8\textwidth}}
    \hline
		\textbf{input} & \texttt{ANY\_0 is an tuple with 3 elements : None , result of method 
                     CLASS\_0 . FUNC\_0 and None .} \\
    \textbf{ref.}  & \texttt{ANY\_0 = None, CLASS\_0.FUNC\_0(), None} \\
    \textbf{pred.} & \texttt{ANY\_0 = None, (None,), CLASS\_0.None} \\
    \textbf{remark}& \texttt{A typical form, XXX.YYY appears in code, whenever the user wants to 
                     express that YYY is an attribute of entity. Here, we hypothesize that the 
                     model assigned ``None'' as CLASS\_0's attribute due to the period at the end 
                     of the sentence.} \\
    \hline
		\textbf{input} & \texttt{call the function CLASS\_0 . FUNC\_0 -LSB- CLASS\_0 . FUNC\_0 -RSB- 
                     with 3 arguments : raw string STR\_0 , ANY\_2 and ANY\_0 without the first 
                     and last element} \\
    \textbf{ref.}  & \texttt{ANY\_1 = CLASS\_0.FUNC\_0('STR\_0', ANY\_2, ANY\_0[1:-1])} \\
    \textbf{pred.} & \texttt{ANY\_0 = CLASS\_0.FUNC\_0('STR\_0', ANY\_0 + 1)} \\ 
    \textbf{remark}& \texttt{We observe that the model typically performs well on short phrases, 
                     but on long phrases it fails to capture all the nuances expressed in 
                     different parts of the sentences. In the following example, the model 
                     completely fails to realise ANY\_0 as a list} \\
    \hline
  \end{tabular}

  \section{Interseting finds}

  \begin{tabular}{ l p{0.8\textwidth}}
    \hline
    \textbf{input} & \texttt{ANY\_0 is an empty dictionary .} \\
    \textbf{ref.}  & \texttt{ANY\_0 = dict()} \\
    \textbf{pred.} & \texttt{ANY\_0 = \{\}} \\
    \textbf{remark}& \texttt{Here, the model has correctly predicted an empty dictionary, but
                     the evaluation has missed this. } \\
    \hline
    \textbf{input} & \texttt{ANY\_0 is an list with an element ANY\_1 .} \\
    \textbf{ref.}  & \texttt{ANY\_0 = [ANY\_1]} \\
    \textbf{pred.} & \texttt{ANY\_0 = [ANY\_1]} \\
    \textbf{remark}& \texttt{This is the only example where the model has correctly predicted
                     on a list based query. We hypothesize that this might be because the 
                     list based queries are generally very long causing the model to forget 
                     long term dependencies in the code.} \\
    \hline
    \textbf{input} & \texttt{substitute ANY\_1 for ANY\_0 , ANY\_4 , ANY\_3 and ANY\_2 , 
                     respectively .} \\
    \textbf{ref.}  & \texttt{ANY\_0, ANY\_4, ANY\_3, ANY\_2 = ANY\_1} \\
    \textbf{pred.} & \texttt{ANY\_0, ANY\_4, ANY\_3, ANY\_2 = ANY\_1} \\
    \textbf{remark}& \texttt{It is interesting to notice that the model is able to understand
                     which part of the code fall on the left and right side of the assignment
                     operator. Also, it correctly places all the variables in their 
                     proper order. } \\
    \hline
  \end{tabular}


\end{document}
