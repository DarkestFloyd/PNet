\documentclass{IEEEtran}
\usepackage{blindtext}
\pagenumbering{gobble}

\title{\textbf{Natural Language to Code \\ \large{Project Proposal}}}

% Nischal Mahaveer Chand
% Varun Sundar Rabindranath
% Sai Krishna Karanan
\author{
    Sai Krishna Karanam 
    \texttt{karanam.s@husky.neu.edu}
    \and \\
    Nischal Mahaveer Chand 
    \texttt{mahaveerchand.n@husky.neu.edu}
    \and \\
    Varun Sundar Rabindranath 
    \texttt{rabindranath.v@husky.neu.edu}
}
\date{}

\begin{document}

    \maketitle

    \section{Introduction}

    The task of converting Natural Language to Code is well established. With recent
    research, we have achieved basic code generation from annotated text and
    code comments. Although the techniques used for the same are state-of-the-art, our
    team feels there are parts that still need improvement.

    Take for example the recent paper by Microsoft \cite{balog2016deepcoder}. The paper 
    describes one such technique where code is generated from the description.
    The above generates only a limited set of machine instructions and is domain specific.
    Our team proposes to create a code generator for a high level language, like Python.
    
    With this project, we aim to reduce development time and cost for building efficient, 
    readable, and reliable software. It should be noted that the project does not aim at 
    providing a framework to build industry standard deployable code, but is rather a 
    proof-of-concept. With further research, the former might be possible. 
    As another use case, artificial code generators can also generate exception routines
    for common exceptions that human programmers sometimes fail to identify, like checking 
    bounds when indexing an array, or checking for zero during dividing.
    With the introduction of a new language everyday, it becomes more and more important
    to establish an interface between natural language and programming languages.

    For this project, our plan is to use the comments in the code to train models, 
    where the comments would act as labels for training and the code as the expected 
    output. Apart from this, we also plan on using data from other sources as described 
    in the Dataset and Evaluation section.
    
    \section{Related Work}
    Techniques to generate regular expressions \cite{kushman2013using}, input parsers 
    \cite{lei2013natural}, 
    UML diagrams, 
    object oriented class layout, and general purpose code from Natural Language (NL) 
    specification have been researched for a long time. In this project we will focus only 
    on generating executable code from NL.

    Recent approaches in converting NL to executable source code predominantly use 
    neural network techniques. These approaches follow two basic patterns. One, is to 
    directly convert NL to source code by modeling the task as a Sequence-to-Sequence 
    conversion task, using Recurrent Neural Networks (RNN) (Lili Mou et al., 2015) 
    \cite{mou2015end}
    to that end. The other is to perform semantic parsing on the NL input and then convert 
    the produced syntax tree to code (Pengcheng Yin et al., 2017) \cite{yin2017syntactic}. 
    However, research \cite{yin2017syntactic} 
    shows that the second approach is better than the first and emphasizes the use of 
    syntax information of the target language, in NL to code tasks.

    We consider the recent research paper by Pengchen Yin et. al. 
    \cite{yin2017syntactic} , on syntactic neural 
    model for code generation as the primary motivation for this project. They propose 
    an approach where they translate NL to an Abstract Syntax Tree (AST) first, thus using 
    the grammar of the target language as a prior knowledge. They report a 10\% absolute 
    improvement in accuracy compared to the previous state-of-the-art on standard datasets. 
    Since, converting from AST to code can be done deterministically, their system always 
    produces syntactically correct, executable code. This aspect is lacking in the 
    Sequence-to-Sequence models, where the correctness of the generated code is not 
    guaranteed. They use a bidirectional LSTM network for encoding each word, in the NL input, 
    as  a context specific embedding and an RNN as a decoder to generate an output AST.

    Maxim Robinovich et. al. \cite{rabinovich2017abstract}, propose an approach similar 
    to \cite{yin2017syntactic} 
    for converting NL to AST. 
    They confirm the experimental results of \cite{yin2017syntactic}, thus again emphasizing
    the use of target 
    language syntax. However Maxim Robinovich et. al., have evaluated their system on a wider 
    range of datasets achieving near state-of-the-art results on average.

    Pengcheg Yin et.al \cite{yin2017syntactic}, report that 25\% of the errors incurred by 
    their system, on one of the datasets, is due to the generated code only partially 
    implementing the required functionality. We hypothesize that this could be due to 
    insufficient training data leading to the encoder-decoder network not seeing enough data 
    to capture certain semantic nuances. 
    Robin Jia et. al. \cite{jia2016data}, propose a novel data recombination technique that 
    induces a high-precision generative model from the training data, and then samples 
    from it to generate 
    new training examples. These new training samples capture important conditional independence 
    properties commonly found in semantic parsing. They report that such a technique beats 
    existing state-of-the-art on one of the datasets they evaluated. We hypothesize that adding 
    this technique to Pencheng Yin et. al’s approach will significantly improve their accuracy.

    \section{Project Plan}
    For this project, we plan to have four different phases,
    \begin{enumerate}
        \item Phase 1: \textit{Literature Survey} \\
            This phase mainly deals with the team reading multiple papers to familiarize ourselves
            with the terminologies and jargons used for the task of converting natural language 
            to code.
        \item Phase 2: \textit{Modeling} \\
            The team will build multiple models for the task. At this moment, we plan to 
            implement Hidden Markov Models, Semantic Parsers, and Multilayer Neural Networks.
        \item Phase 3: \textit{Evaluation} \\
            Each model will be carefully evaluated and examined for potential optimizations.
            The best model will be used for further optimization and improvement.
        \item Phase 4: \textit{Testing} \\
            The selected model will be tested against various data points, finding any weak 
            spots to improve on, or edge cases to cover.
    \end{enumerate}

    Maxim Rabinovich et. al. report that their system fails to handle an imperative description 
    of complex logic. They also report that their model fails to enforce executeability, 
    well-typedness and semantic coherence in the output. Executeability is handled by 
    \cite{yin2017syntactic}, while the other issues present open research opportunities. We 
    wish to investigate each of these issues and hope to discover novel approaches to solve them.

    \section{Dataset and Evaluation}
    We consider two datasets, HearthStone and Django. We find that these two datasets are 
    used widely for NL to code generation tasks.

    HearthStone is a card based computer game. The HearthStone dataset consists of Python classes 
    implementing the cards, based on the card description. HearthStone dataset is considered 
    difficult due to the terse card descriptions from which the card`s class code should be 
    generated. 

    Django is a web development framework written in Python. The Django dataset consists of 
    lines of code in Python manually annotated with NL descriptions. The Django dataset is 
    relatively more general compared to HearthStone dataset, in the sense that it covers a wider 
    variety of programming use cases.

    As is commonly used for NL to code tasks, we will use accuracy, the fraction of correctly 
    generated code samples, and BLEU-4 as our primary and secondary metrics respectively.

    \bibliographystyle{ieeetran}
    \bibliography{proposal}

\end{document}
